\documentclass[12pt,a4paper,twoside,openright]{report}

\usepackage{hegyhati}


\title{ Title }
\author{ Kovács Patrícia \\\ \\Supervisor: Máté Hegyháti, PhD}


\begin{document} 

  \ChapterUnnumbered{Abstract}{Abstract}{Abstract}
  
  
   
  % List of... pretty much everything  
    
    \tableofcontents
    \listoffigures 

  \ChapterUnnumbered{Köszönetnyilvánítás}{Köszönetnyilvánítás}{Thanks}
  \markboth{Köszönetnyilvánítás}{}
    
  \Chapter{Bevezetés}{Bevezetes}

  \Chapter{A program alapjául szolgáló játékok, eszközök bemutatása}{Example}
  \Section{Ötletet adó játékok}{Alapotlet}
  \Section{A Rizikó játék bemutatása}{Riziko}

  \Chapter{Az elkészítendő szoftver specifikációja}{Specifikacio}

  \Chapter{Felhasznált technológiák}{Example}
    \Section{Java}{Java}
    \Section{Lehetséges grafikus könyvtárak összehasonlítása}{Example}
    \Section{A JavaFX ismertetése}{JavaFX}
    \Subsection{Egy JavaFX alkalmazás}{JavaFXPelda}
        
    \Section{OAuth 2.0}{Oauth2}
    \Section{JSON}{JSON}
    \Section{Maven}{Maven}
    \Section{Fejlesztést támogató eszközök}{Example}
    \Subsection{NetBeans IDE}{NetBeans}
    \Subsection{SceneBuilder}{SceneBuilder}
    \Subsection{Fitnesz adatbázisok}{FitneszAb}


  \Chapter{Fejlesztési dokumentáció}{Fejlesztes}
  \Section{Alkalmazás regisztrálása a Strava weboldalán}{AppRegisztralas}
  \Section{Strava adatok elérésének megvalósítása}{Example}
  \Section{Játék logika megvalósítása}{Example}
  \Section{A Strava adatok eléréséért felelős és a játék osztályok kapcsolata}{Kapcsolat}
  \Section{Kinézet szerkesztése}{Example}

  \Chapter{Eredmény bemutatása}{Eredmeny}

  \Chapter{Összefoglalás és további fejlesztési lehetőségek}{Osszefoglalas}
  

  % References
  \addcontentsline{toc}{chapter}{Irodalomjegyzék}
  \bibliography{references}
  \bibliographystyle{plain}

\end{document}
