A program elkészítéséhez egy olyan programozási nyelvet kellett választani, amelyben kényelmesen lehet grafikus felületet készíteni, illetve amiben különösebb nehézség nélkül meg lehet csinálni a webes authentikációt a fitnesz adatbázishoz való csatlakozáshoz. Ezt több technológiahalmaz is támogatja, mint például a C++ és QML StackView vagy a typescript, Node.js és Elektron kombináció, de a Java nyelvet választottam, mivel ezt ismerem a legjobban és minden követelménynek megfelelt, továbbá egy igencsak elterjedt programozási nyelv. 

A Java egy olyan programozási nyelv, mely kikényszeríti az objektumorientált elvek betartását. James Gosling vezetésével a Sun Microsystems egyik csoportja, a JavaSoft tervezte, első verzióját 1996-ban adták ki. 

Tulajdonságai: 

egyszerű, vagyis kevés nyelvi eszközt kínál 

szintaktikájában nincsenek mutatók, garbage collected nyelv 

objektumorientált 

hibatűrő 

biztonságos 

platformfüggetlen, különböző gépeken futtatható 

nincsenek implementációfüggő részei 

több szál párhuzamos futtatását biztosítja 

Azért a Java nyelvet választottam, mert tanulmányaim során ezt ismertem meg legjobban és tovább szerettem volna bővíteni a tudásom a témakörön belül. 