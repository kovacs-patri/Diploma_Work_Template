A Maven egy olyan szoftver, mely segítségével automatizálhatók a build folyamatok, illetve projektek menedzselésére is alkalmazható. Használatával a program felépítéséhez szükséges függőségeket nem szükséges manuálisan letölteni, mivel ezt automatikusan megteszi a projekt fordítása előtt egy repositoryból. Rengeteg függőség elérhető például a Maven Central Repository-ból, de léteznek más repository-k is és magunk is létrehozhatunk sajátot. 

A Maven alapja a Project Object Model (POM), amely egy XML fájl, melybe meg kell adni minden szükséges információt, ami a program felépítéséhez elengedhetetlen. Ez alapján a Maven képes elvégezni a build folyamatot, különböző dokumentációk, riportok generálását, források lefordítását, az alkalmazás összecsomagolását. A build folyamat kimenetele az artifact. Ez például 3rd party library esetén JAR állomány, web alkalmazás esetén pedig WAR állomány. Létre lehet hozni vele még például projekt sablonokat is különböző paraméterek megadásával. 