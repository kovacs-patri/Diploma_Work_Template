A specifikációban leírt program egy része, hogy valamilyen weboldalról adatokat gyűjtsön egy azonosított felhasználó aktivitásáról. 

Lehetséges források, és azokhoz való hozzáférés lehetőségei 

Lehetséges források közül a Google Fit és a Strava API-jával ismerkedtem, mivel ezek széles körben használt eszközök, és biztosítanak Java hozzáférést. Mindkettő esetben REST API segítségével történik az adatokhoz való hozzáférés. 

Ezeken kívül létezik még sok más olyan forrás, ahonnét aktivitási adatokat lehet elérni, mint például a Garmin Connect, vagy a RunKeeper, viszont ezek közül nem mindegyiknek ingyenes és publikus az API-ja, vagy nem feltétlenül van hozzá Java lib. 

A két lehetséges forrás közül a Strava-t választottam, mivel a platformok között talán ez a legnépszerűbb, és több más fitnesz adatokat tároló adatbázisból lehet rá szinkronizálni aktivitási adatokat. Ezen kívül nyílt API-val rendelkezik, és van hozzá Java támogatás is, így ezt ideálisnak tartottam. 