Tulajdonságai: 

a Java natív része 

támogatja több szál párhuzamos futtatását, generikus típusok és lambda kifejezések használatát valamint az adatkötést 

JavaFX kód írható bármilyen JVM által támogatott szkriptnyelven, mint például Groovy vagy Scala 

multimédiás megoldásokat is nyújt (videók és hangok lejátszása) 

a platformon elérhető kodekeket használja 

lehetővé teszi webtartalom beágyazását az alkalmazásba 

támogatja a különféle effektek és animációk alkalmazását, melyek játékfejlesztés szempontjából előnyösek 

A JavaFX platform komponensei 

Ebben az alfejezetben részletesebben bemutatom a 2. ábra alapján, hogy a JavaFX platform milyen komponensekből épül fel. 


A felhasználói felület Scene graph-ból, úgynevezett jelenetgráfból épül fel, a JavaFX publikus API segítségével. A jelenetgráf vizuális elemeket tartalmaz, melyeket Node-oknak, más néven csomópontoknak hívnak és hierarchikus módon, fa-struktúra szerűen vannak rendezve. Ezek az elemek képesek kezelni a felhasználói inputokat, lehetnek effektjeik, transzformációik és állapotaik. Ilyen vizuális elem lehet például egy gomb vagy egy szövegmező.  

A Prism egy hardveres gyorsítású grafikus pipeline, amely a jelenetgráf rendereléséért felel. 

A Glass Windowing Toolkit ablakozási feladatokért felelős, melyeket a natív operációs rendszertől függően jelenít meg, valamint kezeli az eseménysorokat is. JavaFX-ben a fő szálban kerülnek kezelésre az események, ezt a szálat JavaFX Application Thread-nek hívják. A jelenetgráf csak ezen a szálon módosítható. Amikor módosításra kerül egy jelenetgráf, például egy felhasználói input hatására, akkor a Prism-nek újra meg kell jelenítenie a jelenetgráfot.  

A média motor hang és videó lejátszásáért felelős, amihez a platformon lévő kodekeket használja. GStreamerre épül, amely egy nyílt forráskódú multimédia keretrendszer. 

A webmotor webes tartalom jelenetgráfba ágyazásáért felel, melyet a Prism jelenít meg. WebKit alapú, amely egy nyílt forráskódú webböngésző-motor. A Safari böngésző is ezt a motort használja, valamint több alkalmazás is macOS-en, iOS-en és Linuxon is. Továbbá korábban a Google Chrome és az Opera is ezt használta. 

A Quantum Toolkit az előbbiek feletti absztrakciós szint, mely megkönnyíti ezen alacsony szintű komponensek közötti koordinációt. 

Egy JavaFX alkalmazás életciklusa 

JavaFX alkalmazás futtatásakor a következő két fő szál jön létre: 

JavaFX Application Thread 

JavaFX Launcher 

Egy JavaFX alkalmazás az Application osztályból öröklődik. Az alkalmazás életciklusa alatt az Application osztály launch() metódusának meghívásakor a következő metódusok futnak le az alábbi sorrendben: 

paraméter nélküli konstruktor 

init() 

start() 

Az init() metódus a JavaFX Launcher Thread-ben hívódik meg, a start() pedig a JavaFX Application Thread-ben. Amikor az alkalmazás leáll, akkor a stop() metódus hívódik meg, bezárni pedig a Platform.exit() metódussal lehet.  

Model-View-Controller (MVC) és a JavaFX 

A Model-View-Controller, röviden MVC, egy programtervezési minta, amely szétválasztja a megjelenítést (view) az adatoktól (model), melyeket egy vezérlő (controller) kapcsol össze. 

Előnye, hogy bármelyik rész könnyen lecserélhető, így egyszerűbben módosítható. A JavaFX az MVC programtervezési mintát követi. A 3. ábrán látható az MVC működése: 


Model 

A szoftver állapotát leíró változókat, objektumokat tartalmazza. 

View 

Kétféleképpen építhető fel a felhasználói felület egy JavaFX alkalmazásban: 

Java kóddal 

FXML használatával 

Az FXML egy XML alapú, szkriptelhető jelölőnyelv. Segítségével a kódtól teljesen függetlenül, külön fájlban készíthető el a felület. Létezik az FXML-nek vizuális szerkesztője is, ez pedig a Scene Builder. Segítségével gyorsabban és egyszerűbben össze lehet állítani a GUI-t. 

Controller 

Az FXML fájlban megadható, hogy melyik kontroller osztály tartozik az adott felülethez. A kontroller osztály eléri az FXML-ben lévő elemeket és műveleteket végezhet a GUI-n. 