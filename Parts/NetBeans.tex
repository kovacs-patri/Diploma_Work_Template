A NetBeans egy Java nyelven írt, nyílt forráskódú, integrált fejlesztői környezet. Fejlesztése 1996-ban kezdődött egy egyetemi projekt keretén belül, eleinte Xelfi névre hallgatott. Első verziója 1997-ben jelent meg. 1999-ben a Sun Microsystems felvásárolta a programot és megnyitotta annak forráskódját. 2010-ben az Oracle felvásárolta a Sun-t, ezáltal a NetBeans is Oracle fejlesztés alá került. A NetBeans 8.2-es kiadása után, 2016-ban az Oracle átadta az Apache Software Foundation-nek a program forráskódját, így jelenleg ők fejlesztik tovább a fejlesztői környezetet. 

Tulajdonságai: 

ingyenes és nyílt forráskódú 

támogatja a java platformokat 

hibakereső eszközöket nyújt 

integrált támogatást nyújt olyan szkriptnyelvekhez, mint például a Javascript, a PHP vagy a Groovy 

többféle programozási nyelvet is támogat 

Lehetett volna választani ezen kívül más fejlesztői környezetet is, mint például az Eclipse vagy az IntelliJ. Azért esett erre a választásom, mert ingyenes, egyszerűen kezelhető, és már használtam korábban is. 