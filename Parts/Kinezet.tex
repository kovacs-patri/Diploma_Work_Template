A felület kinézetének módosításához a JavaFX Cascading Style Sheets-et használtam. Ez a W3C 2.1-es CSS verzióján alapul. Ez egy stílusleíró nyelv, mely segítségével meghatározhatjuk a felületen megjelenő különböző elemek megjelenését. A JavaFX CSS a CSS olyan kiterjesztéseit tartalmazza, melyek különböző speciális JavaFX szolgáltatásokat is támogatnak. Nem támogatja a CSS „float”, „overflow”, „position” és a „width” elrendezésre vonatkozó tulajdonságait, ezeket a JavaFX kódban kell beállítani. Ezeken kívül, különböző HTML-specifikus elemekhez kapcsolódó beállításokat nem támogatja, mint például a táblák. A JavaFX CSS fájljában a beállítandó tulajdonságok neve elé „-fx-„ írandó. 

A programhoz egy CSS fájlt szerkesztettem, ez tartalmaz minden szükséges beállítást, melyek meg vannak adva a felületen a különféle, megjelenő elemeknek. Itt szerepelnek a betűkre, gombokra, Label-ekre, Slider-re, a háttérre, a ScrollPane-re és a benne lévő lista megjelenéséhez kapcsolódó információk. Hogy ezek megjelenjenek a GUI-n, az FXML fájlban a megfelelő elemek tulajdonságainak felsorolásában szerepelnie kell a használandó CSS fájl nevének a következő módon: stylesheets="@style.css". 

Ha egy elemre külön, a CSS fájl által tartalmazott .class beállítást szeretnénk megadni, azt jelen esetben például a háttér beállítására vonatkozóan a style.css-ben szereplő .background{}-on belüli tulajdonságokat, azt a styleClass="background" módon kell jelölni az FXML fájlban. CSS fájltól függetlenül is lehetőség van megadni az FXML fájlban JavaFX CSS megjelenítési tulajdonságokat például style="-fx-background-color: #b38f6b; -fx-border-color: #5D544C;" módon. 