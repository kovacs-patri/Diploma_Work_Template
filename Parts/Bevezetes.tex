Napjainkban egyre jobban háttérbe szorul a mozgás.
Manapság az emberek sokkal szívesebben töltik az idejüket számítógép előtt ülve, minthogy kimozduljanak a szabadba, ami rendkívül egészségtelen.
Viszont egy olyan számítógépes játékkal, ami megköveteli a mozgást ahhoz, hogy játszani lehessen, rávehetőek lehetnek az emberek arra, hogy kimozduljanak a házból. 

Ehhez a témakörhöz kapcsolódik a Szakdolgozat feladatom, ami a Rizikó nevű társasjátékon alapuló, körökre osztott stratégiai társasjáték fejlesztése Java programozási nyelven.
A játékos célja az ellenfél területeinek elfoglalása saját hadseregének fejlesztésével, katonái mozgatásaival és harcba küldésével.
Egy leegyszerűsített Rizikó megvalósításán túl a fejlesztendő program célja a felhasználó sportos életmódra való ösztönzése.
Ezt úgy éri el, hogy a felhasználónak a játékban történő előrehaladáshoz mozognia kell, melyet a valamilyen fitnesz adatbázisából nyert adatokkal igazol. 

A következő fejezetben bemutatom a program alapjául szolgáló játékokat és eszközöket, ezen belül a Rizikó társasjátékot részletezve, ezt követően ismeretem az elkészítendő szoftver specifikációját és funkcionális céljait.
A továbbiakban a program készítésénél felhasznált technológiákat részletezem, ezen belül bemutatom egy minta Maven-JavaFX projekt felépítését.
Ezután rátérek a fejlesztői dokumentációra, melyben részletesen ismertetem az elkészített program szerkezetét.
Ezt követően bemutatom a lefuttatott programot, majd pedig dolgozatom utolsó fejezetében szót ejtek a továbbfejlesztési lehetőségekről és összefoglalom a munkám. 