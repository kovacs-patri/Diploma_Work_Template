Egy FXML dokumentumot kétféleképpen szerkeszthetünk, az egyik mód, hogy a szerkesztőjében kézzel írunk be mindent, vagy pedig egy vizuális szerkesztő segítségével, mely nem más, mint a Scene Builder. A Scene Builder egy nyílt forráskódú, felhasználói felület tervező eszköz. Segítségével gyorsan össze lehet állítani a fejlesztendő program kinézetét, „fogd és vidd” módszerrel egyszerűen rendezhetőek a megjeleníteni kívánt elemek, melyeknek minden tulajdonságát beállíthatjuk a Scene Builder-en belül. Az ebben megalkotott kinézet kódja automatikusan létrejön a háttérben. 