Az OAuth 2.0 egy olyan keretrendszer, amely bizonyos HTTP szolgáltatásokhoz, felhasználói fiókok adataira vonatkozóan nyújt korlátozott jogosultságot harmadik féltől származó alkalmazások számára. 

Működése 

A 14. ábrán látható az OAuth 2.0 működése: 

Az alkalmazás engedélyt kér a felhasználótól az adatai eléréséhez 

Ha elfogadásra kerül a kérés, az alkalmazás megkapja az engedélyt 

Elküldi az authorizációért felelős szervernek a saját identitására vonatkozó hitelességét, valamint a felhasználótól megkapott engedélyt, egy Access Token-ért cserébe 

Ha a szerver érvényesnek ítélte a kérést, válaszul elküldi az Access Token-t 

Az alkalmazás az erőforrásokért felelős szervernek elküldi az Access Token-t, a szükséges erőforrásokért cserébe 

Ha érvényes a Token, a szerver válaszul elküldi a szükséges adatokat 