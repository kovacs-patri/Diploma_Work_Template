A JSON (JavaScript Object Notation) egy embereknek is könnyen írható és olvasható, hierarchikus felépítésű adatcsere-formátum. Részben a JavaScript nyelven alapul. XML alternatívaként használják, leggyakrabban egy webalkalmazás és egy szerver közti adatok átvitelére. 

Egy példa alapján bemutatom a felépítését. 

{"cars": [{ 

"brand": "Ford", 

"model": "Mustang", 

"year": 2021, 

"available": false, 

"owner": "John" 

        }, { 

"brand": "Audi", 

"model": "A7", 

"year": 2015, 

"available": true, 

"owner": null 

        } 

]} 

Egy objektumra vonatkozó adatok { és } írandók. A példában egy "cars" nevű objektum található, mely egy tömbbel rendelkezik, ami két objektumot tartalmaz. A tömb elemei [ és ] közé írandók. Mint látható, egy objektum több név-érték párt is tartalmazhat, ezeket vesszővel kell elválasztani. A szöveges értékeket idézőjelek közé kell írni. A példában is látszik, hogy sztringeken kívül értékül adható még szám, logikai igaz vagy hamis, illetve null érték is. 