Mivel két kontroller osztály található meg a projektben, biztosítani kell ezek egymással való kommunikációját. Ehhez a játék kontroller osztályában létrehoztam egy metódust, mely beállítja az osztályhoz azt a Controller objektumot, ami a metódus meghívásakor megadásra kerül benne. 

A Controller osztályban pedig a következőképp állítottam ezt be egy olyan metóduson belül, mely a játék felületének megnyitásáért felel: 


Lehetőség van az FXML fájl betöltéséért felelős FXMLLoader objektumnak lekérni a kontrollerét, ezen keresztül állítottam be a játék kontroller osztályához a bejelentkezésért és a felhasználó adataiért felelős kontrollert. Ezen kívül ebben a metódusban átadásra kerülnek a játék kontrollerének a betöltendő mentésre, valamint az új játékra vonatkozó adatok, melyet a GameEngine példánya megkap paraméterben és így feltöltésre kerülnek a MapController-ben a Country objektumok. 

Ezután ellenőrzésre kerül, hogy érvényes-e még a felhasználóhoz tartozó korábban eltárolt Access Token, ha nem, akkor ez frissül, majd megtörténik a felhasználó aktivitási listájának frissítése és ehhez kapcsolódóan a pontok beállítása attól függően, hogy új játék kezdődött-e, vagy korábbi mentés került betöltésre. A MapController ez alapján megkapja a megfelelő kezdeti értékeket, melyeket az indításkor meg is jelenít a felületen. 

Mivel a felhasználó aktivitási adatai csak a játék erősítési fázisában számítanak, így a MapControllerben a StackPane-ekre vonatkozó eseménykezelőben a REINFORCE ágban hozzáadásra kerültek különböző, pontok beállításáért felelős elágazások, melyek azt vizsgálják, hogy van-e egyáltalán a felhasználónak feltöltött aktivitása a Strava weboldalán, ha igen, akkor csak egy van-e, vagy több, és ezek szerint számolja a felhasználó pontjait, ebből következően pedig az erősítési fázisban lerakható maximális katonák számát állítja. Amint egy katona elhelyezésre került a REINFORCE fázisban, módosításra kerülnek a felhasználó elköltött és elérhető pontjai. Minden 1000 méter megtétele után plusz egy lehelyezhető katona jár az erősítési fázisban. Ha nem történt aktivitás a belépés óta, akkor csak egy katonát tehet le ebben a szakaszban. 

A játék felületén található egy frissítés gomb, ami a játékos körében bármikor elérhető, ez frissíti az Access Token-t, ha az lejárt, és ha történt aktivitás, akkor kiszámolja a megfelelő pontszámot, ennek alapján beállítja a játékos számára elérhető katonák számát és a megfelelő értékek megjelennek a GUI-n. A JSON fájlba írásért és az ezekből való olvasásért a WriteAndReadJSON osztályban található függvények felelnek. 

A fájlba írás során öt JSONArray-be történik az adatok kiírása egy JSONObject-en belül. Az első JSONArray-ben a Country objektumokra vonatkozó adatok kerülnek tárolásra, úgy, mint az id, a birtokos és a megyén található katonák száma. A másodikban a játékmenet aktuális helyzetére vonatkozó információk vannak tárolva, vagyis hogy hányadik kör van éppen, melyik fázis, mennyi a letehető katonák száma, melyik játékos köre zajlik épp. A harmadik JSONArray a felhasználóra vonatkozó adatokat tartalmazza, melyekből a User objektum értékei töltődnek fel. A játékos neve, id-je, utolsó aktivitásának az id-je, az Access és Refresh Token-je, az Access Token lejárati ideje, az eddig megtett összes útja, az elköltött és felhasználható pontjai, a játékba való első belépésekor lévő utolsó aktivitásának id-je tartozik ide. A negyedik JSONArray-en a Strava app adatai találhatók meg, vagyis a clientId és a clientSecret. Tárolásra kerül még a játékban történő eddigi dobásokra vonatkozó lista is. 

A játék különböző pontjain történik automatikus mentés, a gép köre után, és a felhasználó fázisainak elején. A felületen megjelenő kilépés gomb megnyomásának hatására is történik mentés. Egy felhasználó csak egy mentéssel rendelkezik, a már létező mentés törlésre kerül új mentés hatására. A mentett fájl neve tartalmazza a felhasználó id-jét. 