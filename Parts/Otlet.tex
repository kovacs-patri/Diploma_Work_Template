\section{Ötletet adó játékok}

A számítógépes és mobil játékok különböző módszerekkel motiválják a felhasználókat a játékra.
Általában minél többet játszik valaki, annál gyorsabban, vagy annál nagyobb eséllyel kaphat jutalmakat.
Ilyen például a szintlépési rendszernél, hogy bizonyos szinteket elérve új funkciók érhetők el a játékban, vagy egyes mérföldköveket elérve különféle jelvényeket kaphat a felhasználó.

Léteznek olyan, általában mobil játékok, melyekben a továbbhaladáshoz, valamint a jutalmak megszerzéséhez elengedhetetlen, hogy a felhasználó kimozduljon a szabadba.
Ilyen például a Pokémon GO, a Jurassic World Alive és az Ingress mobiltelefonos játékok.
Ezek elérik azt, hogy a felhasználóik többet mozogjanak a szabadban, de eközben elég gyakran a mobiltelefonra kerül a játékosok figyelme, emiatt sok baleset is származott az ilyen játékokból.
A Marylandi Egyetem elvégzett egy kutatást arra vonatkozóan, hogy a gyaloglás közbeni mobiltelefonhasználat okozta baleseteket összegyűjtse.
A kutatásban a 2010 és 2011 közötti eseteket vizsgálják, melyek az Egyesült Államokban történtek.
310 ilyen esetet találtak, és ez a szám azóta évről-évre folyamatosan nő, mely az NHTSA statisztikáiból is látszik.
 
Ez szolgált alapjául az általam elkészített programnak, mely egy asztali alkalmazás, ami mozgásért cserébe jutalmaz.
A játék Magyarország térképén zajlik, amihez a Honfoglaló adott ötletet. 

Azonban nem csak játékok léteznek a felhasználók mozgásra való motiválásához.
Vannak olyan alkalmazások, melyek pénzszerzési lehetőséget nyújtanak a mozgásért cserébe.
Ilyenek voltak korábban például az Achievemint, mely minden 25.000 pontért cserébe \$25-t kínált, vagy a GymPact, aminél meg lehetett adni, hogy a következő héten hány alkalommal szeretne a felhasználó edzeni, és ha teljesítette, egy kevés pénzt nyert azoktól a felhasználóktól, akik nem teljesítették a heti edzéseket.
Jelenleg ilyen alkalmazások például a Sweatcoin, a StepBet és az Achievement.
Vannak azonban olyan alkalmazások is, melyek nem pénzzel jutalmaznak, hanem például különböző kedvezményekkel, vagy a pontokat jótékonysági szervezetek számára lehet felhasználni.
Ilyen például a Charity Miles és a BetterPoints.


\section{A Rizikó játék bemutatása}

A Rizikó egy stratégiai területhódító társasjáték, melynek célja, hogy a játékosok a térképen lévő lehető legtöbb területet meghódítsák, emellett teljesítsék a feladat-kártyájukon kitűzött feladatot.

\Figure {riziko}{Rizikó társasjáték}{width=15cm}

\subsection*{Tartozékok}
\begin{itemize}
\item 1 db játékterep 
\item 14 db feladat-kártya 
\item 44 db Rizikó-kártya 
\item 6 játékfigura készlet 
\item 5 dobókocka 
\end{itemize}

A játéktáblán a 6 kontinens térképe látható, összesen 42 területre osztva.
A kontinensek különböző színűek és 4-12 terület található rajtuk.

A hadseregből 6 teljes szett van, mindegyik három típusú bábut tartalmaz:
\begin{itemize}
\item gyalogság (1 értékű) 
\item lovagság (5 gyalogság értékű) 
\item tüzérség (10 gyalogság, vagy 2 lovagság értékű)
\end{itemize}

Kártyák: 44 db Rizikó-kártyát tartalmaz a játék, melyeken a területek képei vannak, rajtuk gyalogság, lovagság vagy tüzérség alakkal.
A feladat-kártyákon különböző célok találhatók, melyek elérésével lehet megnyerni a játékot.

\subsection*{Játékszabály}

A játék kezdetén a játékosok színt választanak.
A feladat-kártyák közül mindenki kap egyet, amelyen szereplő feladatot titokban kell tartani.
A játék legelején, a területszétosztási fázisban a Rizikó-kártyák kerülnek szétosztásra, amik azt jelölik, hogy a játék kezdetén mely játékos melyik területeket birtokolja.
Ezekre a területekre mindegyik félnek legalább egy hadosztályt el kell helyeznie.
Ha ez megtörtént, egy játékos összegyűjti az összes Rizikó-kártyát, majd megkeverve, lefordítva leteszi.
Ezt követően megkezdődik a háború fázis, amely körökre van osztva és minden körnek három szakasza van.

\subsection*{Első szakasz}

A játék első szakasza az erősítés.
Egy forduló kezdetekor a soron lévő játékos erősítést kérhet.
Annyit kaphat, amennyi az általa birtokolt területek számának egyharmada, de legalább hármat.
Bónusz erősítést kaphat abban az esetben, ha legalább egy teljes kontinenst birtokol.
A játékos becserélheti birtokolt Rizikó-kártyáit hadosztályokra, ha az előző körében sikerült meghódítania legalább 1 területet.
Egy fordulóban akármennyi kártya becserélhető, de ez nem kötelező, kivéve, ha minimum 5 kártyával rendelkezik.
Ebben az esetben az egész kártyakészletét be kell cserélnie.
A becserélt kártyák visszakerülnek a pakliba.

\subsection*{Második szakasz}

A második szakasz a csata.
Támadás csak határos, vagy szaggatott vonallal összekötött területek között indítható.
A támadó félnek arról a területről, amelyről támad, legalább 2 hadosztályt kell felhasználnia a támadáshoz, vagyis ezen a területen egy védőnek, és legalább egy támadónak kell lennie.
A játékosok 1, 2 vagy 3 hadosztállyal támadhatnak, de 1 védőt hátra kell hagyniuk a területen.

A csata kockadobásokkal zajlik.
A támadó fél piros kockákkal dob, a védő kékekkel.
Ha a védőnek 3-nál kevesebb hadosztálya van a megtámadott területen, abban az esetben csak 1 kockával dobhat.
A dobott számok összehasonlításra kerülnek.
Csökkenő sorrendbe rendezzük a kockákat, a nagyobb számot dobó fél győz, a vesztes pedig levesz egy hadosztályt az adott területről és a fel nem használt játékfigurái közé teszi.
Ha több kockával történtek a dobások, akkor a második legmagasabb érték dönti el, hogy melyik fél veszti el a következő hadosztályát.
Egyenlő értékek esetén a védő fél nyeri a csatát.
Ha a védekező félnek a csata során elfogynak az adott területen lévő hadosztályai, a támadónak el kell foglalnia ezt a körzetet az utolsó támadásában felhasznált hadosztályaival. 

Ezután a játékos eldöntheti, hogy szeretne-e még több területet megtámadni, vagy a fordulójának következő szakaszába lép.
Ha valamelyik játékos összes hadosztálya megsemmisült a támadások során, az a fél elvesztette a játékot.
Ha legalább egy terület elfoglalásra került, akkor az adott játékos kap egy kártyát.

\subsection*{Harmadik szakasz}

A csata után az átcsoportosítás szakasz következik.
A játékos átcsoportosíthatja egyetlen hadosztályát egyik területéről egy másik, ezzel szomszédos területére úgy, hogy legalább egy hadosztályt hátrahagy.
Ha ezt megtette, a következő játékos köre következik.

\subsection*{A játék nyertese} 

Az a játékos, aki elsőként teljesítette a feladat-kártyáján lévő feladatot, bemutatja a kártyáját a többi játékosnak és megnyerte a játékot.
Két játékos esetén a feladat-kártyák nem kerülnek kiosztásra, ebben az esetben az a fél nyer, akinek elsőként sikerül megsemmisíteni az ellenfél hadosztályait.