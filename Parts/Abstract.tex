One of the biggest problems of today’s generation is the lack of physical exercise.
People prefer to spend their time sitting in front of their computer instead of going out into the nature.
However, they can be encouraged to lead a more active lifestyle with a game that requires walking, running or other form of exercise to progress in the game.

There are several software solutions in various application stores that require steps, active minutes, etc. from the user to progress or obtain various rewards.
Their appearance propagated active lifestyle among users, however, more and more accidents result from people paying more attention to their mobile phones while walking than to traffic and potential hazards.

This problem was the basis of the my project, which is a desktop application that rewards movement in return.
It encourages users to go outdoors to walk, run, bike, or even play sports, as long as distances can be covered.
Playing and exercising are done separately, so users do not lose attention to the environment when it is important.

The result of my dissertation is a software written in Java based on a board game called Risk.
The player's goal is to occupy the opponent's areas with the help of his army.
In exchange for exercise, the user gets bonus soldiers at one stage of the game, so they have a better chance of defeating their opponent.
The application accesses player activities from the Strava database, which has 76 million users.
In my dissertation I present the developed application with the applied technologies, its detailed structure, and the presentation of the completed application.