One of the biggest problems of today’s generation is a lack of exercise.
People prefer to spend their time sitting in front of a computer rather than moving out into the nature.
However, they can be encouraged to lead a more active lifestyle with a game that requires movement to progress in the game.
There are already several typically solved solutions in various application stores in which it is essential for progress and obtaining various rewards for the user to go out to move.
Using them increases user activity, but more and more accidents result from that people pay more attention to their mobile phones while walking than to traffic and potential hazards.
This problem was the basis of the project I created, which is a desktop application that rewards movement in return.
It encourages users to go outdoors to walk, run, bike, or even play sports as soon as distances can be covered, while not losing one’s attention to the environment.
The result of my dissertation is a software written in Java based on a board game called Risk.
The player's goal is to occupy the opponent's areas with the help of his own army.
In exchange for movement, the user gets bonus soldiers at one stage of the game, so they have a better chance of defeating their opponent.
The application accesses player activities from the Strava database, which has 76 million users.
In my dissertation I present the completed software with the technologies which I used, detailing its structure and the presentation of the completed application.