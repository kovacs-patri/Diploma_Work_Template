A feladat célja egy olyan asztali alkalmazás és játék fejlesztése, mellyel egy felhasználó tud a gép ellen játszani.
A cél Magyarország összes megyéjének elfoglalása, melynek ötlete a Honfoglalóból jött, a csata dinamikája pedig a Rizikó társasjátékon alapul azzal a csavarral, hogy az erősítési fázisban sportteljesítményhez kötött a lehelyezhető katonák száma.
Ebben a fejezetben ennek a feladatnak a specifikációját fejtem ki részletesebben.

Az elkészítendő szoftver egy Rizikó társasjátékra épülő társasjáték fejlesztése, mely valamilyen fitnesz  adatbázisából nyer adatokat, amik alapján korlátozás történik a játékos köreinek erősítési fázisban lehelyezhető katonák számára.

A program egy egyjátékos módot támogató Desktop alkalmazás.
A felhasználó a gép ellen játszik, a térkép pedig Magyarország térképe.
A programnak el kell érnie egy fitnesz adatbázist, ahonnét hozzáfér a felhasználó aktivitási adataihoz.
A felhasználónak a játék kezdetétől megtett távolságai alapján korlátozásra kerül, hogy mennyi katona elérhető számára a játék erősítési szakaszában.
Mivel, ha a játékos keveset sportol, úgy csak egy katonát tud felhasználni az erősítési szakaszban, így előfordulhat, hogy sokáig elhúzódik a játék, ezért a   felhasználónak lehetősége van elmenteni a játék állását, és egy későbbi program indítás során ez betölthető.

A játék logikája a Rizikó társasjáték leegyszerűsített változata.
Nem tartalmaz kártyákat, a területek a játék kezdetekor véletlenszerűen sorsolódnak ki.
Először a játékos kezd, aki a körének első szakaszában erősítést küld egy területére.
Amint ezzel végzett, elkezdődik a támadási fázis, itt a játékos addig támad, amíg akar, vagy ameddig tud, ezután átcsoportosítja valamennyi katonáját egyik területéről a másikra, majd átadja a kört a gépnek.
A gép véletlenszerűen választ egy területet, melyre erősítést küld, ezután megkezdi a támadást.
Az a fél nyer, aki a térkép összes területét birtokába veszi.

Az előbb leírtakon kívül fontos követelmény még, hogy átlátható és könnyen használható grafikus felülettel rendelkezzen, valamint az is, hogy a kód legyen verziókezelve.