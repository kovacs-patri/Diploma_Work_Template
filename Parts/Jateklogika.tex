Magához a játék megvalósításához öt Java osztályt és egy hozzá tartozó FXML fájt készítettem. A Java osztályok név szerint a következők: 

Country 

CountryDiceRoll 

GameEngine 

MapFXMLController 

Map 

A következőkben ezen osztályok tartalmát részletezem. 

A Country osztály tartalmazza a megyékre vonatkozó információkat, úgy, mint az id-je, mely egy egyedi azonosító, magában foglalja, hogy hány katona van az adott területen, annak birtokosát, és mivel körökkel jelöltem meg a felületen a megyéket, így a hozzá tartozó, javafx.scene.shape csomag által nyújtott kört is itt tárolom, valamint az ezen megjelenő szöveget is, mely a területen tartózkodó katonák számát jeleníti meg. 

Ebben az osztályban találhatók még az ezeket lekérő, illetve beállító függvények és metódusok, valamint különböző függvények arra vonatkozóan, hogy eldöntésre kerüljön, hogy két megye ugyanannak a játékosnak a birtokában van-e, hogy az adott területről hány katonával támadhat és hánnyal védekezhet, továbbá egy másik terület megtámadására vonatkozó függvény. 

A területről támadó katonák maximális száma 3, a védekezőé pedig 2. A támadás kockadobásokkal történik, amelyért felelős függvények a CountryDiceRoll osztályban találhatók meg. Itt kerül tárolásra a támadó és a védekező fél kockadobásainak listája, a legnagyobb dobás mindkét fél részéről, valamint, hogy hány oldalú a dobókocka és az ezeket beállító, illetve lekérő függvények és metódusok. A kockadobások csökkenő sorrendbe rendezéséért is ennek az osztálynak egy függvénye felel. 

A Country osztály tehát a támadásra vonatkozó metódusában ezt használja annak eldöntésére, hogy egy támadás során melyik fél veszít katonát. Egy támadás során vagy az egyik, vagy a másik fél elveszít egy katonát. Amelyik fél nagyobbat dob, az nyer, az ellenféltől pedig levonásra kerül egy katona. Ha megegyezik a két legnagyobb dobás, abban az esetben a védekező fél nyeri a csatát, vagyis a támadó elveszít egy katonát. 

A Map osztály egy HashMap-et tartalmaz a megyék szomszédosságaira vonatkozóan, melynek feltöltése egy JSON fájlból történik. A HashMap tartalmát visszaadó függvényen kívül még a szomszédosságra vonatkozó függvényeket tartalmaz, úgy, mint annak ellenőrzése, hogy két adott megye szomszédos-e, egy megye szomszédsági listája tartalmazza-e a paraméterben megadott megyét, valamint egy adott megye szomszédsági listájának visszaadására vonatkozó függvényt. 

A GameEngine nevű osztály a játékmenetre vonatkozó dolgokat foglalja magában. Tartalmazza a megyék kapcsolatait, valamint azok Country objektumokkal feltöltött listáját, a játékosok listáját, azt, hogy a játékban eddig ki, melyik megyével melyik megyét támadta és mik voltak a dobások, a játék fázisokat, hogy hány katonát kaphatnak a játékosok az adott körben az erősítési fázisban, jelenleg hányadik körben zajlik a játék és a nyertest. 

Magában foglalja még az ezeket beállító és lekérő függvényeket, illetve metódusokat. Itt történik még a Country objektumok létrehozása véletlenszerűen vagy fájlból betöltve, a játék fázisok állítása, az ezektől függő, a felületen megjelenő szövegek állítása, a játékosok léptetése az adott kör végén, ellenőrzések arra vonatkozóan, hogy az adott megye meg tudja-e támadni a kiválasztott ellenfelet, vagy sem, a szomszédos megyék közül két megye szövetségesek-e vagy ellenfelek, annak ellenőrzése, hogy van-e már nyertes, majd ennek beállítása. A következőkben ezeket fogom részletesebben leírni. 

Az osztály konstruktorában paraméterként vár egy boolean és egy String változót arra vonatkozóan, hogy új játék betöltésére kerül-e sor, vagy nem és ha nem, akkor milyen nevű fájlt töltsön be. Feltöltésre kerül a kapcsolati lista. A játékos és a gép hozzáadása a játékos listához is itt történik. Új játék indítása esetén egy olyan metódus hívódik meg, amely a kapcsolati listán végigmenve létrehozza az ehhez tartozó Country objektumokat a megfelelő id-vel, 10 és 15 közötti véletlenszerű katonaszámmal, melyet a java.util.concurrent.ThreadLocalRandom csomag ThreadLocalRandom.current().nextInt() függvényével számoltam. 

Amennyiben nem új játék indítására kerülne a sor, hanem már létező mentés betöltésére, úgy egy json fájlból olvasó metódus kerül lefutásra. Ez a konstruktorban megadott String változó nevét kéri paraméterként, ezzel a névvel ellátott fájlt próbálja meg beolvasni. A fájl három JSONArray listájából nyeri ki a szükséges adatokat. Az első lista a megyékre vonatkozik, innét kapja meg a megyék id-jét, a rajtuk lévő katonák számát és a birtokosukat. Egy for ciklusban erre vonatkozóan feltöltésre kerül a megyék listája. A következő vizsgált JSONArray a játékban eddig történő kockadobásokra vonatkozik. Ismét egy for ciklus segítségével zajlik az eddigi dobási lista feltöltése, úgy, mint a játékállás feltöltésénél. Itt a játék aktuális fázisa, az erősítési fázisban elérhető katonák száma, az aktuális játékos id-je kerül feltöltésre, valamint az, hogy hányadik körben áll a játék. 

A játék fázisai enum-okkal vannak jelölve. A fázisok sorban a következők: 

REINFORCE 

SELECT_REGION 

SELECT_ENEM 

ATTACK 

MOVE_SELECT_FIRST 

MOVE_SELECT_SECOND 

A REINFORCE az erősítési fázist jelzi, itt történik az adott körben felhasználható katonák hozzáadása a saját területekhez, egyesével. A SELECT_REGION az erősítést követő, azt a saját terület kiválasztását jelöli, amivel támadni szeretne a játékos, a SELECT_ENEMY azt jelzi, hogy ellenfél választása következik épp, ATTACK fázisba lép, amint támadás történt és amíg zajlik. Ezt követően következik a MOVE_SELECT_FIRST, amely a saját katonák egy másik, saját, szomszédos területre való átcsoportosítását jelzi, annak is azt a részét, amikor olyan területet kell kijelölni, amelyről át szeretnénk vinni a katonákat máshová. A MOVE_SELECT_SECOND pedig az utolsó fázis egy játékos körében, amikor ki kell választani, hogy hová szeretnénk áthelyezni a katonákat. 

A kör végén megtörténik a felhasználható katonák számának beállítása, az aktuális játékos a következő játékosra való léptetése, valamint a REINFORCE fázis beállítása. A különböző fázisokban történő megyékkel kapcsolatos ellenőrzések, valamint a támadással kapcsolatos metódus is ebben az osztályban találhatók meg. 

Két megye akkor ellenfél, ha különböző a birtokosuk. Egy megye csak akkor tud megtámadni egy másikat, ha szomszédok és különböző játékos birtokolja őket. Két megye akkor szövetséges, ha ugyanaz a játékos birtokolja őket. Külön függvényben szerepel, ha ők emellett szomszédok is. 

A GameEngine osztály tartalmaz még egy megye elfoglalására vonatkozó metódust, mely átállítja az elfoglalt terület birtoklójának nevét a támadó nevére, valamint a két megye katona számát is beállítja. Az elfoglalt területen annyi katona lesz, amennyivel a támadó megtámadta, a támadó területén pedig ennyivel kevesebb. A játék nyertesének eldöntése pedig annak ellenőrzésével történik, hogy a játékosoknak van-e még olyan területe, amelyről támadni tud, ha valamelyiknek nincs már ilyen, akkor vége a játéknak. 

Végül pedig a MapFXMLController osztály tartalmazza az FXML fájlban lévő komponensekre vonatkozó események irányításáért és a felület változásáért felelős metódusokat, valamint itt zajlik a játékmenet irányítása, az ehhez kapcsolódó függvények is itt találhatók. A következőkben ezeket részletezem. 

Az osztály magában foglal a GameEngine osztályból egy példányt, az aktuálisan kiválasztott területekhez kapcsolódó változókat, a gép választható területeire vonatkozó listát, illetve az aktuálisan kiválasztott terület szomszédossági listáit a szövetségesekre, valamint az ellenfelekre vonatkozóan. Az FXML dokumentumban megfogalmazott ArrayList objektumot a térképen lévő köröket tartalmazó StackPane elemekhez kapcsolódóan, valamint több más, FXML objektumot, melyek valamilyen formában módosításra kerülnek a program futása során. 

Ezeken kívül tartalmaz még a Strava bejelentkezéséért felelős Controller osztályából és a WriteAndReadJSON osztályból egy példányt, továbbá az új játékra és a mentésre vonatkozó változókat, de ezeket majd egy későbbi pontban fejtem ki, amikor a két kontroller osztály kapcsolatáról lesz szó. 

A start() metódusban történik meg a GameEngine példány inicializálása, beállításra kerülnek a StackPane-ekre vonatkozó tulajdonságok, úgy, mint a hozzájuk tartozó megye, a feliratuk, a körük, valamint az eseménykezelők, mely a StackPane-ekre történő egérkattintásra hívódik meg. Itt zajlik még a javafx.scene.control csomagja által biztosított Tooltip objektum beállítása minden egyes StackPane-re, mely arra szolgál, hogy a kurzort fölé húzva megjelenítsen az egérmutató mellett egy szöveget, ami ebben az esetben a megyék teljes nevei, melyek a GameEngine osztály egyik függvényéből kerülnek kinyerésre id alapján. Ezután ebben a metódusban a GUI frissítése is megtörténik, illetve a felületen megjelenő különböző Label objektumok értékeinek beállítása is. 

A GUI frissítéséért felelős függvény az alapértelmezettre állítja a Label-ek értékeit, emellett a köröket is beállítja egy másik metódusban meghatározott alap színükre, amely a megyék tulajdonosai szerint állítja be a színeket. A StackPane-ekre beállított eseménykezelő az azokra való egérkattintások hatására lép működésbe. Elsősorban azt ellenőrzi, hogy a játékos körében történt-e a kattintás. Ha nem, akkor nem történik semmi, ha pedig igen, akkor switch-case ágak segítségével ellenőrzi a GameEngine osztálytól lekért aktuális fázist, és ettől függően más-más működés lép életbe a StackPane-ekre kattintva. 

Ha a REINFORCE fázisban történt a kattintás, ellenőrzésre kerül, hogy a játékosnak vannak-e elérhető katonái a GameEngine osztályban, melyeket elhelyezhet ebben a fázisban a saját területein. Ha igen, akkor ezek száma csökken 1-el, és ez hozzáadásra kerül az adott StackPane-hez tartozó Country objektum katonák számát tároló változójának értékéhez. Amint a lehelyezhető katonák száma eléri a nullát, vagy a fázis lezáró gombra kattintás történik, a REINFORCE fázis véget ér, és SELECT_REGION fázisba lép a program, majd az elérhető katonák száma megváltoztatásra kerül olyan értékre, mely a későbbiekben a gép számára lesz felhasználható. 

A SELECT_REGION állapotban bekövetkező kattintás hatására egy olyan metódus hívódik meg, ami ellenőrzi, hogy megfelelő, vagyis saját megyére történt-e a kattintás. Amennyiben igen, tárolásra kerül a kijelölt terület id-je, és a jelenlegi fázistól függően a választott terület által megtámadható területek listájához hozzáadásra kerülnek az ehhez elérhető megyék egyedi azonosítói, mely a GameEngine osztály erre vonatkozó függvényével kerül ellenőrzésre. 

Ennek vizuális szemléltetésére egy olyan metódus meghívására kerül sor, mely a körök színeit, átlátszóságát, valamint a rajtuk elhelyezett javafx.scene.effect csomag által biztosított Effect-et módosítja. Amennyiben a kiválasztott területről lehetséges a támadás, vagyis több, mint 1 katona van rajta, illetve SELECT_REGION vagy SELECT_ENEMY állapotban van a játék, a támadható szomszédok köreinek színe pirosra állítódik és Lighting Effect kerül rájuk, az összes többi átlátszósága felére csökken, a kiválasztott körét kivéve. 

Ha nem megfelelő StackPane-re történt a kattintás, akkor a játék marad SELECT_REGION fázisban. Helyes terület kiválasztása után a SELECT_ENEMY fázisra váltás következik. Ebben az állapotban történő megyékre kattintáskor aktuális fázistól függően ellenőrzésre kerül, hogy a kiválasztott terület jelen esetben támadható-e a korábban kijelölt területről, vagy sem. Amennyiben nem, úgy visszaáll a játékfázis SELECT_REGION állapotba és frissítésre kerül a GUI a fázishoz tartozó alap állapotba. Ha pedig igen, úgy beállításra kerül a többi, támadható szomszéd átlátszósága világosabbra, csak a kijelölt ellenfél, valamint a korábban kiválasztott terület körének színe marad átlátszatlan, a jobb átláthatóság kedvéért, valamint a felületen megjelenítésre kerül egy Pane a támadásra vonatkozóan. Ez három gombot tartalmaz. 

Az egyik az ellenfél terület egyszeri megtámadására vonatkozik, a másik ennek addig történő támadására, amíg az lehetséges, a harmadik pedig a támadási szándék visszavonására. Az egyszeri támadás gomb megnyomására ATTACK fázisba lép a játék, és a GameEngine támadásra vonatkozó függvényét meghívva lezajlanak a kockadobások és a vesztes fél elveszít egy katonát. 

A csatában résztvevő mindkét megye adatai frissítésre kerülnek, mely megjelenik a felületen is. A játék aktuális köre, a támadó fél neve, a megye neve, amelyről a támadás történt, az ehhez tartozó legnagyobb kockadobás, valamint a védő terület neve, illetve a hozzá tartozó legnagyobb dobás hozzáadásra kerül egy kockadobási listára, mely a játékban történő eddigi összes támadás ezen adatait tárolja. Ez a lista megjelenik a felületen egy gomb megnyomásának hatására, mely a felhasználó és a gép általi dobásokat is tartalmazza. 

Ha a támadás során az ellenfélnek már nem marad katonája, az elfoglalásra kerül a GameEngine osztály erre vonatkozó metódusának meghívásával. Ha a játékos támadó területén már csak egy katona maradt, akkor a támadás abbamarad. A játék állapota mindkét eset után SELECT_REGION-ra változik, a felület frissítésre kerül a fázisnak megfelelően, a körök az alapértelmezett színükre változnak az új adatokkal. A gombnyomás hatására ellenőrzésre kerül az is, hogy van-e a játéknak nyertese. 

A folyamatos támadásra vonatkozó gomb megnyomására addig történik automatikusan támadás, amíg az ellenfél területén van katona, vagy pedig a támadó fél területén egynél több katona található. Ha elfogytak az ellenfél katonái a megtámadott megyén, akkor az elfoglalásra kerül és ismét SELECT_REGION fázisba kerül a játék, a GUI frissül. Itt is ellenőrzésre kerül, hogy nyert-e közben valamelyik fél. 

A harmadik gomb megnyomása csak a kiválasztott ellenfél megye támadási szándékának visszavonására vonatkozik, megnyomására alaphelyzetbe áll a felület és ismét egy saját terület kiválasztására kerül sor. 

A StackPane-ekre vonatkozó eseménykezelőben még a MOVE_SELECT_FIRST és a MOVE_SELECT_SECOND fázisokra történik még ellenőrzés. Az előbbi állapotba a fázis lezáró gomb megnyomásának hatására van lehetőség átlépni. ATTACK, SELECT_REGION, SELECT_ENEMY fázisokban történő megnyomásakor MOVE_SELECT_FIRST állapot lesz az aktuális. 

Ez a játékos egy területéről, egy másik, szomszédos, saját területére való katonák átcsoportosításának kezdetét jelenti. Itt kell kiválasztani, hogy mely területről szeretnénk mozgatni. Ugyan azt a metódust használja, mint a korábban kifejtett saját terület választásért felelős case ág, de a jelenlegi fázisnak megfelelően nem a szomszédos ellenfelek listáját tölti fel, hanem a szomszédos szövetségesekét, illetve nem pirosra színezi ezeket a köröket, hanem zöldre, de ezeken kívül ugyan úgy működik. 

Megfelelő terület kiválasztására MOVE_SELECT_SECOND fázisba lép a játék, mely alatt StackPane-re történő kattintás során szintén azt a metódust hívja meg, mint amit a SELECT_ENEMY ág, de a mozgatással kapcsolatos ellenőrzések történnek rajta. Ha olyan terület került kijelölésre, melyre lehetséges az átcsoportosítás, akkor megjelenik egy Pane, melyen ki lehet választani az átcsoportosítandó katonák számát. A JavaFX Slider elemét használtam ennek az adatnak a bevitelére. Átcsoportosítás csak olyan területről lehetséges, amelyen több, mint egy katona van, csak ilyenkor jelenik meg a korábban említett ehhez kapcsolódó Pane. A területen lévő katonák közül egyet mindenképpen ott kell hagyni, ezért a Slider-en kiválasztható maximális érték az adott megyén lévő katonák számánál egyel kevesebb, a minimálisan választható pedig egy. 

A Slider-re beállítottam egy Listener-t, ami annak aktuális értékének változását figyeli, és ez az érték mindig frissen jelenik meg a Slider alatt elhelyezett Label-en. Az említett Pane tartalmaz egy átcsoportosítás megerősítésére vonatkozó és egy visszalépésre gombot. Előbbi megnyomásának hatására az elsőnek kijelölt megye katona számából levonásra kerül a Slider-en kiválasztott mennyiségű katona, a másodiknak kijelölt terület katona mennyiségéhez pedig ez hozzáadódik, a GUI frissítésre kerül, és elkezdődik a gép köre. A visszalépésre vonatkozó gomb megnyomására a fázis ismét MOVE_SELECT_FIRST lesz, és az adott fázishoz tartozóan alaphelyzetbe áll a felület. A fázis lezáró gomb megnyomásának hatására átugorható az átcsoportosítás és megkezdődik a gép köre. A játékos körében mindegyik fázisban, ha a térképre történik kattintás a StackPane-eken kívül, akkor a kör kijelölések visszavonásra kerülnek, és a megfelelő fázisba lép a játék. 

Annak érdekében, hogy láthatóak legyenek a gép lépései, egy új szálat hoztam létre a gép körére vonatkozó esemény metódusán belül, melynek végrehajtása a program bizonyos pontjain szüneteltetésre kerül a java.lang.Thread csomagja által nyújtott Thread.sleep() metódus segítségével, ezzel hasonló hatást elérve, mintha a gép is úgy választaná ki a megyét, mint egy felhasználó. Az új szál létrehozása miatt azokat a programrészeket, melyek a szálon belül módosításokat végeznek a GUI-n, JavaFX-ben a Platform.runLater() metódusba kell beleírni, hogy megfelelően működjön a program, és helyesen megjelenítse a felületen történő változásokat. 

Hasonlóképpen, mint a játékos körének állapotainál, ebben az esetben is a GameEngine osztály játékfázisaira vonatkozó Enum objektumai kerülnek vizsgálatra switch-case ágakkal. Először beállításra kerül, hogy bizonyos gombnyomási funkciók ne váljanak elérhetővé a felhasználó számára. Ezután a REINFORCE fázisban a gép által birtokolt megyék id-je egy listába kerül eltárolásra, melyből véletlenszerűen kiválasztásra kerül egy, amire a felhasználható katonák kiosztásra kerülnek egyesével. Ha ez megtörtént, létrejön egy olyan lista, melyre azok a gép által birtokolt területek kerülnek hozzáadásra, amelyekről támadni is tud. Ezek közül véletlen számú területtel támad a körének támadási szakaszában. Az állapot SELECT_REGION-re kerül beállításra. Ezen belül egy területének kiválasztásához egy listára kerül minden olyan StackPane, mely a gép azon területeit jelöli, amin elegendő katona van a támadáshoz, illetve van neki olyan szomszédja, melyet meg tud támadni. Ez a lista minden alkalommal kiürül, mielőtt ilyen területek gyűjtésére kerülne sor, mert a játék további állapotaiban változás történik a területekre vonatkozóan. Ha ez a lista üres marad, akkor a gép nem tud támadni, ellenőrzésre kerül, hogy nyert-e valaki, és a játék állapota MOVE_SELECT_FIRST-re változik. Ha nem maradt üres a lista, akkor ezek közül a Math.random() függvény segítségével véletlenszerűen választ egyet, melyet átszínez, majd a játék a SELECT_ENEMY fázisba kerül. 

Itt ugyanaz a metódus kerül meghívásra, mint ennél az állapotnál a felhasználó esetében, de mivel paraméterben jelölve van, hogy melyik fél hívja meg, így ez alapján, ha a gép által kerül meghívásra, feltöltésre kerül egy lista, mely tartalmazza a gép által választott terület támadható szomszédjait. Ezek közül kiválasztásra kerül egy terület, mely a felhasználó ellenfélválasztásához hasonló módon átszíneződik, majd a program átlép a támadási szakaszba. Addig történnek a gép támadásai, amíg lehetősége van rá. Ez annyiszor játszódik így le, amennyi az erősítés végén generált random szám. Közben ellenőrzés történik arra, hogy van-e a játéknak nyertese. 

Amint ez lejátszódott, elkezdődik az átcsoportosítási szakasz. Itt kiürítésre kerül a gép olyan köreinek listája, melyről támadást tud indítani, és feltöltődik azon köreivel, ahonnan át tud csoportosítani katonákat. Ezek közül véletlenszerűen kiválasztásra kerül egy megye, melynek köre átszíneződik. Ha nem tud ilyen területet választani, ellenőrzésre kerül, hogy nyert-e valamelyik fél és a játékos köre következik. Amennyiben lehetősége van az átcsoportosításra, úgy a kiválasztott területről véletlen számú katonát visz át egy szomszédos megyére, úgy, hogy minimum egyet hátrahagy. Ezután a játékos következik. Ekkor ismét elérhetővé válnak a korábban letiltott gombok a felhasználó számára és új kör indul. 