A mai generáció egyik legnagyobb problémája a mozgáshiány.
Az emberek szívesebben töltik az idejüket a számítógép előtt ülve, ahelyett, hogy kimozdulnának a természetbe.
Azonban aktívabb életstílusra ösztönözheti őket egy olyan játék, ami megköveteli a mozgást a játékban való továbbhaladáshoz. 

A különböző alkalmazásboltokban található már több olyan, tipikusan játék megoldás, melyekben az előrehaladáshoz, illetve különféle jutalmak megszerzéséhez elengedhetetlen, hogy a felhasználó kimenjen a szabadba mozogni.
Ezek használatával a felhasználók aktivitása nő, ugyanakkor egyre több baleset származik abból, hogy az emberek közlekedés közben jobban figyelnek a mobiltelefonjukra, mint a forgalomra és a lehetséges veszélyforrásokra. 

Ez a probléma szolgált alapjául az általam elkészített programnak, amely egy olyan asztali alkalmazás, ami mozgásért cserébe jutalmaz.
Arra ösztönzi a felhasználókat, hogy kimenjenek a szabadba sétálni, futni, biciklizni, vagy bármilyen sportot űzni, amivel távolságokat lehet megtenni, és eközben nem veszi el az ember figyelmét a környezetről. 

A szakdolgozati munkám eredménye egy olyan Java nyelven írt program, mely a Rizikó nevű társasjátékon alapszik.
A játékos célja az ellenfél területeinek elfoglalása saját serege segítségével.
A felhasználó mozgásért cserébe bónusz katonákat kap a játék egy fázisában, így nagyobb eséllyel győzheti le az ellenfelet.
A játékos aktivitásaira vonatkozó adatokat a 76 millió felhasználóval rendelkező Strava adatbázisból éri el a program. 

Dolgozatomban bemutatom az elkészült programhoz felhasznált technológiákat, részletesen bemutatom annak felépítését, valamint bemutatom az elkészült alkalmazást. 